%Authored by Victor Ekekrantz, Stockholm
%Started writing 2021-12-08
%victoriee.eth.link

%Stored in the following repository https://github.com/VictorieeMan/TheNewMersennePrimeConjecture

\documentclass[12pt,a4paper]{article}
\usepackage[utf8]{inputenc}
\usepackage[english]{babel}
\usepackage{amsmath}
\usepackage{systeme}
\usepackage[thmmarks, thref]{ntheorem}
\usepackage{amsfonts}
\usepackage{amssymb}
\usepackage{fancyhdr}
\pagestyle{fancy}
\lhead{Victor Ekekrantz}
\chead{Compiled\\\today\ kl. \currenttime}
\rhead{Page \thepage}
\cfoot{Stockholm}
\usepackage{tikz}
\usetikzlibrary{arrows,matrix,positioning}
\usepackage{gauss}
\usepackage[yyyymmdd]{datetime}
	\renewcommand{\dateseparator}{--}
\usepackage{hyperref}
\usepackage{gauss}
\usepackage{cite}
%https://divyanshikathuria27.wordpress.com/2015/02/09/gauss-sty-package-for-operations-on-matrix/
%https://mirrors.rit.edu/CTAN/macros/latex/contrib/gauss/gauss-doc.pdf
%\rowops || \colops
%\mult{}{}
%\add[]{}{}
%\swap{}{}
%$$
%\begin{gmatrix}[p]
% 1 & 0 & -1 & 1 \\
% 2 & 1 & -1 & 0 \\
% 3 & 1 & -2 & 1 \\
% -1 & -1 & 1 & 0 \\
% 1 & 2 & 2 & -4
% \colops
% \mult{0}{v_1}
% \mult{1}{v_2}
% \mult{2}{v_3}
% \mult{3}{v_4}
%\end{gmatrix}
%$$

%Variables
\newcommand{\Subject}{}
\newcommand{\TaskNr}{}

\setlength\parindent{0pt}

\author{Victor Ekekrantz}
\title{The New Mersenne Prime Conjecture}

\makeindex

\begin{document}
\maketitle
\newpage
\tableofcontents
\newpage

\section{Acknowledgements}
Bateman, P. T., J. L. Selfridge, and S. S. Wagstaff. “The Editor’s Corner: The New Mersenne Conjecture.” The American Mathematical Monthly 96, no. 2 (1989): 125–28. \url{https://doi.org/10.2307/2323195}.

\section*{The New Mersenne Prime Conjecture}
Let $p$ be any odd number. If two following conditions hold, then so does the third:
\begin{itemize}
	\item[a)] $p=2^k\pm 1$ or $p=4^k\pm 3$ for some natural number $k$.
	\item[b)] $2^p - 1$ is prime (a Mersenne prime)
	\item[c)] $(2^p+1)/3$ is prime (a Wagstaff prime)
\end{itemize}

\section{Introduction}
This mathematical paper looks into proving The New Mersenne Prime Conjecture. Will it succeed? Nobody knows, but it might be some interesting findings along the way. And a lot of prime numbers are guaranteed.


\bibliography{citations}{}

\end{document}